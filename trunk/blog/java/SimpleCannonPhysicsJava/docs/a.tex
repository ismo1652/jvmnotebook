\documentclass{article}

\usepackage{algorithmic}
\usepackage{amsmath}

\oddsidemargin = .1in
\textwidth = 6in
\textheight = 7.5in
\headheight = 0in
\headsep = 0in
\topmargin = .1in
\floatsep = .1in
\textfloatsep = \floatsep
\intextsep = \floatsep

\begin{document}
{\Large

%% From: http://sip.clarku.edu/tutorials/TeX/intro.html
%% Also see: http://www.thestudentroom.co.uk/wiki/LaTex
%% http://en.wikibooks.org/wiki/LaTeX/Mathematics


Here you can write the x-components for the projectile:
\begin{equation}
  v_x = v_mx = v_m cos(\theta_x)
\end{equation}

\begin{equation}
  x = v_xt = (v_m cos(\theta_x))t
\end{equation}

Here is the y-component of displacement:
\begin{equation}
  a_y = -g
\end{equation}
\begin{equation}
  v_y = v_{my} + at = (v_m cos(\theta_y)) - gt
\end{equation}

\begin{equation}
y_o = y_b + L\: cos(\alpha)
\end{equation}

\begin{equation}
y = y_o = v_{my}t + (1/2)at^2
\end{equation}

\begin{equation}
y = (y_b + L\: cos(\alpha) + v_m cos(\theta_y)t - (1/2)gt^2
\end{equation}

}

\end{document}
