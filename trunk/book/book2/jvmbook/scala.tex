%%--------------------------------------
%% Scala and Lift
%%--------------------------------------
\chapter{JVM Languages - Scala}

Scala was created in 2001 and is currently at version
2.7.5 as of 6/3/2009. Scala is a stark contrast to Clojure. David Pollak had this to say about Scala's type system, "Put another way, Scala is statically typed with optional duck typing... the opposite of Objective-C et. al. with their duck typing with optional static typing." Clojure on the other hand is dynamic in the sense that you don't have to explicitly define a type whenever you need to use a particular object, but you can define the type specification using type hints.


Our web-application would not be complete without a clear approach 
for persisting the link data. So we have used the Hibernate ORM 
(object relational mapping) library do the backend persistance work for us. 
It is not really necessary to use Hibernate for such a simple 
application, but as your enterprise application grows, 
the need for a more robust persistance mechanism will greatly become evident. 
MySQL 5.0.2 is used for our database and most of the recent 
MySQL connector APIs will work with this example.

Almost like Struts, a lot of the hibernate settings 
are defined in a hibernate configuration file, 'hibernate.cfg.xml' 
and your hibernate mapping file, 'Botlist.hbm.xml'. 
Normally the most important settings for your application 
include what database dialect you are using; we are using MySQL 
and the definition of your hibernate POJO beans. 
The simple bean contains an almost one-to-one mapping between 
your database fields and the Java members, accompanied by 
the appropriate getters and setters.