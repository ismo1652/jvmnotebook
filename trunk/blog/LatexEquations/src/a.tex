\documentclass{article}

\usepackage{algorithmic}

\oddsidemargin = .1in
\textwidth = 6in
\textheight = 7.5in
\headheight = 0in
\headsep = 0in
\topmargin = .1in
\floatsep = .1in
\textfloatsep = \floatsep
\intextsep = \floatsep

\begin{document}
%%\renewcommand{\normalsize}{\fontsize{14}{16}\selectfont}
%%\renewcommand{\normalsize}{\fontsize{16}{18}\selectfont}
%%\renewcommand{\mbox}{\fontsize{16}{18}\selectfont}
{\Large
\begin{displaymath}
 \int H(x,x')\psi(x')dx' = -\frac{\hbar^2}{2m}\frac{d^2}{dx^2}
                          \psi(x)+V(x)\psi(x)
\end{displaymath}

\begin{equation}
 2 + 2 = 4
\end{equation}

\begin{equation}
  50 apples \times 100 apples = lots of apples 
\end{equation}

\begin{equation}
 I = \! \int_{-\infty}^\infty f(x)\,dx \label{eq:fine}.
\end{equation}

%% From: http://sip.clarku.edu/tutorials/TeX/intro.html
%% Also see: http://www.thestudentroom.co.uk/wiki/LaTex

Let us see how easy it is to write equations.
\begin{equation}
 \Delta =\sum_{i=1}^N w_i (x_i - \bar{x})^2 .
\end{equation}

It is a good idea to number equations, but we can have a
equation without a number by writing
\begin{equation}
 P(x) = \frac{x - a}{b - a} , \nonumber
\end{equation}

and then 

\begin{equation}
 g = \frac{1}{2} \sqrt{2\pi} . \nonumber
\end{equation}

\begin{equation}
  x \longleftarrow u
  y \longleftarrow 0
\end{equation}

\begin{algorithmic}

  \FOR{$i = 1 \to 10$} 
    \STATE $i \gets i + 1$
  \ENDFOR 
\end{algorithmic}

}

\end{document}
