\chapter{Overview of the Java Programming Language}

\section {History of Java}
Empty

When most computer users upload a profile image
from their desktop to Facebook's website they don't 
stop to think about the simple binary math rules that are fundamental
to most digital devices. 
We realize that 4 gigabytes of RAM is more memory than 
512 megabytes but we don't visualize the logic chips 
that are involved in an xor 0x100, eax operation for a 
32-bit CISC processor. Software developers have to 
consider memory management or how a computer's 
operating system loads their programs into memory. They don't normally
consider VHDL logic circuit designs, the data paths, arithmetic logic
units or the millions of transistors that make up a modern CPU. Those
low-level details have been intentionally hidden from the user
application developer. The modern CPU may have changed dramatically
over the last decade but at the heart of early digital computing were
simple Boolean operations. These simple rules were combined together
and logic replicated to load programs into memory and then
execute. The rules that control most digital devices are based on
elementary Boolean rules. Cellular automata has a similar bottom-up
approach, rules consist of simple programs (as Stephen Wolfram calls
them) that apply to a set of cells on a grid.

onway's Game of Life cellular automaton is one of the most prominent examples of cellular automata theory. The one dimensional program consists of a cell grid typically with several dozen or more rows and similar number of columns. Each cell on the grid has an on or off Boolean state. Every cell on the grid survives or dies to the next generation depending on the game of life rules. If there are too many neighbors surrounding a cell then the cell dies due to overcrowding. If there is only one neighbor cell, the base cell dies due to under-population. Activity on a particular cell is not interesting but when you run the entire system for many generations, a group of patterns begin to form.
