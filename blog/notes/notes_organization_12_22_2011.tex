\documentclass{article}

\usepackage{algorithmic}
\usepackage{amsmath}

\oddsidemargin = .1in
\textwidth = 6in
\textheight = 7.5in
\headheight = 0in
\headsep = 0in
\topmargin = .1in
\floatsep = .1in
\textfloatsep = \floatsep
\intextsep = \floatsep

\begin{document}

%% From: http://sip.clarku.edu/tutorials/TeX/intro.html
\begin{center}
Berlin Brown (misc research)
\end{center}

\subsection{brain}
At the heart of this theory is how patterns
are formed  in complex systems.

\subsection{patterns}
One of the goals of this chapter is to show
patterns in general emerge in a self-organized fashion.
without any agent-like entity ordering the elements,
telling them when and where to go.

I intend to show that principles of self-organization
like behind all structure or pattern formation, and later,
that the brain itself is an active, dynamic, self-organizing
system.

-- J.A. Scott Kelso, Dynamic Patterns: the self-organization of brain and behavior

\subsection{what is a pattern}

A pattern is a combination of qualities, acts, tendencies etc.
forming a consisent or characteristic arrangment.


\vspace{5 mm}
Test2.



\end{document}
