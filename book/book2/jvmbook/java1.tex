\chapter{ Introduction to Programming with Java }

\section{Spring MVC Framework}

Spring is one of the latest Java Enterprise Edition (J2EE) replacement 
frameworks that is supposed to provide minimal presentation layer support, 
business logic support as well as the backend, persistance support. 
It is basically a framework to bind all frameworks. And, on top 
of everything I just mentioned, it is lightweight. 
From working with it, it is a lot easier to work with than with the early EJBs.

It was created by Rod Johnson, orignally in the year 2000, but recently released over the last few years.
It is a Java framework that is growing rapidly. Recently, BEA have been in talks with
Interface21 (Spring's Host company) for a possible partnership. The key is light but flexible. What does
that mean exactly? Spring is modular, all the Spring libraries are broken up so that you can work with
separate parts of your application individually as opposed to tackling everything at once.

With a MVC(Model-View-Controller) application, you are concerned with modular design,
seperating the View from the Controller and Action logic, but also you need to worry about "Flow". "If
the user submits a 'print' command, seperate that logic from the 'add' logic, and what pages are
presented to the user". Doing this within the limitations of JSP, Servlets, and Taglibs becomes difficult.
So, This reated a need for frameworks like the Pico container and Spring, Struts. 

\begin{lstlisting}
public class Java {
  public static void main(final String [] args)  {
   System.out.println("Java");
  }
}
\end{lstlisting}

We discussed earlier how Jython is basically used for the backend 
coding, that includes communicating with Hibernate. 
Here are the code snippets associated with each of those operations. 
Most of the code is fairly intuitive; at the heart of the 
create operation, you must get the Hibernate SessionFactory 
and initiate a transaction. Once that is done, 
create an instance of the Hibernate POJO bean and populate 
the bean with the data from the Struts ActionForm. 
Once that is taken care of, use the session and transaction 
object to save the data. The Edit operation probably contains 
the most code and is seperated into two Jython classes.

\subsection{Hibernate Object Relational Mapping and Model Beans}

Our web-application would not be complete without a clear approach 
for persisting the link data. So we have used the Hibernate ORM 
(object relational mapping) library do the backend persistance work for us. 
It is not really necessary to use Hibernate for such a simple 
application, but as your enterprise application grows, 
the need for a more robust persistance mechanism will greatly become evident. 
MySQL 5.0.2 is used for our database and most of the recent 
MySQL connector APIs will work with this example.

Almost like Struts, a lot of the hibernate settings 
are defined in a hibernate configuration file, 'hibernate.cfg.xml' 
and your hibernate mapping file, 'Botlist.hbm.xml'. 
Normally the most important settings for your application 
include what database dialect you are using; we are using MySQL 
and the definition of your hibernate POJO beans. 
The simple bean contains an almost one-to-one mapping between 
your database fields and the Java members, accompanied by 
the appropriate getters and setters.

\subsection{Full Text Search with Lucene}

\subsection{Developer tools, Eclipse and Emacs}


